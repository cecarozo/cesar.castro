\documentclass{article}\usepackage[]{graphicx}\usepackage[]{color}
%% maxwidth is the original width if it is less than linewidth
%% otherwise use linewidth (to make sure the graphics do not exceed the margin)
\makeatletter
\def\maxwidth{ %
  \ifdim\Gin@nat@width>\linewidth
    \linewidth
  \else
    \Gin@nat@width
  \fi
}
\makeatother

\definecolor{fgcolor}{rgb}{0.345, 0.345, 0.345}
\newcommand{\hlnum}[1]{\textcolor[rgb]{0.686,0.059,0.569}{#1}}%
\newcommand{\hlstr}[1]{\textcolor[rgb]{0.192,0.494,0.8}{#1}}%
\newcommand{\hlcom}[1]{\textcolor[rgb]{0.678,0.584,0.686}{\textit{#1}}}%
\newcommand{\hlopt}[1]{\textcolor[rgb]{0,0,0}{#1}}%
\newcommand{\hlstd}[1]{\textcolor[rgb]{0.345,0.345,0.345}{#1}}%
\newcommand{\hlkwa}[1]{\textcolor[rgb]{0.161,0.373,0.58}{\textbf{#1}}}%
\newcommand{\hlkwb}[1]{\textcolor[rgb]{0.69,0.353,0.396}{#1}}%
\newcommand{\hlkwc}[1]{\textcolor[rgb]{0.333,0.667,0.333}{#1}}%
\newcommand{\hlkwd}[1]{\textcolor[rgb]{0.737,0.353,0.396}{\textbf{#1}}}%
\let\hlipl\hlkwb

\usepackage{framed}
\makeatletter
\newenvironment{kframe}{%
 \def\at@end@of@kframe{}%
 \ifinner\ifhmode%
  \def\at@end@of@kframe{\end{minipage}}%
  \begin{minipage}{\columnwidth}%
 \fi\fi%
 \def\FrameCommand##1{\hskip\@totalleftmargin \hskip-\fboxsep
 \colorbox{shadecolor}{##1}\hskip-\fboxsep
     % There is no \\@totalrightmargin, so:
     \hskip-\linewidth \hskip-\@totalleftmargin \hskip\columnwidth}%
 \MakeFramed {\advance\hsize-\width
   \@totalleftmargin\z@ \linewidth\hsize
   \@setminipage}}%
 {\par\unskip\endMakeFramed%
 \at@end@of@kframe}
\makeatother

\definecolor{shadecolor}{rgb}{.97, .97, .97}
\definecolor{messagecolor}{rgb}{0, 0, 0}
\definecolor{warningcolor}{rgb}{1, 0, 1}
\definecolor{errorcolor}{rgb}{1, 0, 0}
\newenvironment{knitrout}{}{} % an empty environment to be redefined in TeX

\usepackage{alltt}              % [12pt]
\usepackage[top=0.8in, bottom=0.8in, left=0.5in, right=0.6in]{geometry}

\pagenumbering{gobble}
\usepackage[utf8]{inputenc}
\usepackage{hyperref}
\usepackage[english]{babel}
\usepackage[authoryear]{natbib} 
\usepackage{eurosym}
\usepackage{amsmath}        % split equations
\usepackage{tablefootnote}
\usepackage{float} 
\usepackage{authblk}
\usepackage{array}
\usepackage{placeins}
\usepackage{amssymb}                 % additional math symbols
\usepackage{booktabs}
\usepackage{rotating}
\usepackage{appendix}
\usepackage{setspace}
\usepackage{bm}
\IfFileExists{upquote.sty}{\usepackage{upquote}}{}
\begin{document}
\today{}
\\
\\
\textbf{BI Norwegian Business School (Department of Economics)}\\
\textsc\textbf{Department of Economics}\\
\\
\textbf{Research Statement}\\
\textbf{César Castro Rozo}\\
\\


Based on my previous research, I expect to work in at least two related issues:

\begin{enumerate}
  \item Investigating the (negative) time-varying relationship between oil price ($O_t$) changes and exchange rates ($ER_t$) in the euro area.\\
  It is observed a negative relationship between the price of Brent crude oil (expressed in U.S. dollars) and the bilateral exchange rates euro/U.S. dollar (USD/\euro) from the early 2000s onwards. From a theoretical standpoint, authors such as \cite{Golub1983} and \cite{Krugman1983} highlighted that after an increase in $O_t$ there is a wealth transfer from oil-importing economies to oil-exporting countries and consequently, it is expected a depreciation of the domestic currency with respect to USD in oil-importing economies. On the other hand, authors as \cite{Blomberg1995} show the reversed mechanism through which $ER_t$ changes can affect $O_t$ on the basis of the law of one price for tradable goods. The differences in the direction of the causality are related, among others, to three key issues: (i) the frequency of the data; (ii) the oil-dependence of the country; and (iii) the period of analysis.\\
  This paper will contribute to better understand the dynamic interaction between exchange rates $ER_t$ and oil price $O_t$ in the euro area. In doing so, we consider a Time-Varying Parameter VAR model with the use of daily data from 1987 to 2017 (7520 observations). Over this period, the euro area has experienced major changes in world oil market conditions, including a high role of global demand and its linkage with financial markets. We postulate that the weak negative relationship observed before the early 2000s was led by exchange rate shocks, but the sharp link found later has been driven by oil price shocks. We analyze the potential differences of the effects of $O_t$ shocks on $ER_t$ with a particular focus on the different origin of the $O_t$ movements (supply, global or specific demand shocks, building on the analysis in \citealp{Kilian2009}) and the role of financial markets after the early 2000s and the likely appearance of structural breaks. Based on the previous literature (e.g., \citealp{Granger1969}; \citealp{Primiceri2005}) we want to follow a two-stage approach in order to identify structural changes in the correlation between $O_t$ and $ER_t$ and the changes in the direction of the causality over time.\footnote{We have obtained some preliminary results using bvarsv package in the \textbf{\textsf{R}} programming language (\citealp{Krueger2015}).}
  
  \item Evaluating the sensitivity of inflation in the 19 euro area (EA) members to alternative scenarios about future oil price (OP). Based on this sensitivity we want to assess the consequences of the common monetary policy (CMP) implemented by the European Central Bank (ECB) on inflation convergence and price competitiveness among EA members.\\
  In this paper, we build on our earlier work \citep{Castro2016} assessing the effects of oil price shocks on the inflation and competitiveness among EA members, taking into account the role of the common monetary policy reaction.\\
  As a way to cope with the higher inflation expected after, for example, positive OP shocks (OP increases), the ECB may choose to raise its key interest rate, aggravating the economic downturn caused by direct and indirect effects of OP shocks such as the reduction in real household income (consumers have to pay higher energy bills) or a relative decline in manufacturing activity. Key in this research is the possibility –so far empirically unaccounted by the existing literature– that economic risks may arise due to the inability of the ECB to respond to differential country-specific effects of OP shocks. Indeed, the common monetary policy implemented by the ECB to anticipate the negative effects of OP shocks on inflation, responds to an average inflation in the EA, while there are differential effects of such shocks on inflation in EA members. Consequently, a CMP generates a situation where the key interest rate is too high for countries with low inflation and too low for countries with high inflation.\footnote{This could be related with the literature about asymmetries in the responses of single EA countries after common monetary shock (see, e.g., \citealp{Barigozzi2014}; \citealp{Cavallo2015}).}
  In order to examine this issue we (i) extent the assess of the effects on inflation under different OP scenarios for each EA member country; (ii) analyze the optimal systematic monetary policy for each EA state and the necessity of more active national policies (fiscal policy, structural reform or regulation) to offset the effects of OP shocks; and (iii) evaluate cluster of members according to their levels of inflation convergence.
  Thus, this investigation seek to establish if rather than fight inflationary effects, common monetary policy has helped to push up divergence in the inflation of the euro area members.
\end{enumerate}

\bibliographystyle{apa}
\bibliography{ThesisBib}
\end{document}
