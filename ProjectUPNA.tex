\documentclass{article}\usepackage[]{graphicx}\usepackage[]{color}
%% maxwidth is the original width if it is less than linewidth
%% otherwise use linewidth (to make sure the graphics do not exceed the margin)
\makeatletter
\def\maxwidth{ %
  \ifdim\Gin@nat@width>\linewidth
    \linewidth
  \else
    \Gin@nat@width
  \fi
}
\makeatother

\definecolor{fgcolor}{rgb}{0.345, 0.345, 0.345}
\newcommand{\hlnum}[1]{\textcolor[rgb]{0.686,0.059,0.569}{#1}}%
\newcommand{\hlstr}[1]{\textcolor[rgb]{0.192,0.494,0.8}{#1}}%
\newcommand{\hlcom}[1]{\textcolor[rgb]{0.678,0.584,0.686}{\textit{#1}}}%
\newcommand{\hlopt}[1]{\textcolor[rgb]{0,0,0}{#1}}%
\newcommand{\hlstd}[1]{\textcolor[rgb]{0.345,0.345,0.345}{#1}}%
\newcommand{\hlkwa}[1]{\textcolor[rgb]{0.161,0.373,0.58}{\textbf{#1}}}%
\newcommand{\hlkwb}[1]{\textcolor[rgb]{0.69,0.353,0.396}{#1}}%
\newcommand{\hlkwc}[1]{\textcolor[rgb]{0.333,0.667,0.333}{#1}}%
\newcommand{\hlkwd}[1]{\textcolor[rgb]{0.737,0.353,0.396}{\textbf{#1}}}%
\let\hlipl\hlkwb

\usepackage{framed}
\makeatletter
\newenvironment{kframe}{%
 \def\at@end@of@kframe{}%
 \ifinner\ifhmode%
  \def\at@end@of@kframe{\end{minipage}}%
  \begin{minipage}{\columnwidth}%
 \fi\fi%
 \def\FrameCommand##1{\hskip\@totalleftmargin \hskip-\fboxsep
 \colorbox{shadecolor}{##1}\hskip-\fboxsep
     % There is no \\@totalrightmargin, so:
     \hskip-\linewidth \hskip-\@totalleftmargin \hskip\columnwidth}%
 \MakeFramed {\advance\hsize-\width
   \@totalleftmargin\z@ \linewidth\hsize
   \@setminipage}}%
 {\par\unskip\endMakeFramed%
 \at@end@of@kframe}
\makeatother

\definecolor{shadecolor}{rgb}{.97, .97, .97}
\definecolor{messagecolor}{rgb}{0, 0, 0}
\definecolor{warningcolor}{rgb}{1, 0, 1}
\definecolor{errorcolor}{rgb}{1, 0, 0}
\newenvironment{knitrout}{}{} % an empty environment to be redefined in TeX

\usepackage{alltt}

\usepackage[utf8]{inputenc}
\usepackage[english,spanish]{babel}
\usepackage[authoryear]{natbib}
\usepackage{fancyhdr}                % encabezados y pies
\usepackage{indentfirst}             % sangrías
\pagestyle{empty}
\usepackage{colortbl, xcolor}
\usepackage{hyperref}
\usepackage{eurosym}
\usepackage{amsmath}
\usepackage{tablefootnote}
\usepackage{float} 
\usepackage{authblk}
\usepackage{placeins}
\usepackage{amssymb}                 % additional math symbols
\usepackage{booktabs}
\usepackage{array}
\newcolumntype{L}[1]{>{\raggedright\arraybackslash}m{#1}}   % for table notes
\usepackage{appendix}
\usepackage{setspace} 
\setlength{\parskip}{1em}            % paragraph separation
\setlength{\parindent}{0pt}          % no indent
\IfFileExists{upquote.sty}{\usepackage{upquote}}{}
\begin{document}

\begin{center}
\textbf{UNIVERSIDAD PÚBLICA DE NAVARRA}\\
Facultad de Ciencias Económicas y Empresariales
\vspace{10pt}

Primer concurso ordinario de contratación de profesorado curso 2018/2019\\
\begin{tabular}{l l}
Plaza:  & Nº 4991\\
Categoría: & Profesor Ayudante Doctor\\
Departamento: & Economía\\
Área de Conocimiento: & Fundamentos del Análisis Económico\\
\vspace{10pt}
\end{tabular}

\textcolor[RGB]{85,87,89}{\rule{\linewidth}{0.4pt}}

\textbf{Proyecto investigador}\\
César Castro Rozo\\
Puesto que no se especifica el idioma en el que presentar el proyecto investigador y el idioma científico-académico es el inglés, se presenta el proyecto en inglés.\\
Mayo de 2018\\
\vspace{5pt}
\textbf{``Oil price shocks and monetary policy in the euro area''}\\
\end{center}

\title{Oil price shocks and monetary policy in the euro area}
\author{César Castro Rozo\thanks{E-mail: ccastrorozo@gmail.com}}

\maketitle

\begin{abstract}
The common monetary policy implemented by the European Central Bank (ECB) since 1999 has responded to differential effects of several oil price shocks on inflation of its members. The objective of this research is to evaluate the sensitivity of inflation in the 19 euro area (EA) members to alternative scenarios about future oil price and to assess the consequences of the common monetary policy on inflation convergence and price competitiveness. Countries with inflation far from the ECB target may need a more active national policy to offset the oil price shocks, in the absence of a country-specific monetary policy reaction. Therefore, quantifying the sensitivity of inflation to oil price shocks in the euro area and its members would improve the effectiveness of common monetary policy and national economic policies.\\
\\
  \textbf{Keywords.} Inflation, Oil price shocks, Euro area\\
  \textbf{JEL classification:} C32, E23, E31, Q43
\end{abstract}

\vspace{5pt}

The effects of oil price shocks on macroeconomic variables have been widely studied in the literature since the end of the 1970s. The complexity and relevance of these shocks on the economic behavior and the appropriate policy responses to them have continued to be matter of increasing concern, despite the fact that circumstances have changed (the smaller share of oil in production, the adoption of energy-efficient technologies, the development of alternative energy sources and the improvements in monetary policy, among others).

On the basis of the early work of \cite{Castro2016}, the objective of this investigation is twofold: (i) to quantify the sensitivity of inflation rates in the euro area (EA) and each of its members to alternative scenarios of oil price changes;\footnote{Scenarios can be based on future (global or oil-specific) demand and supply conditions, i.e., surplus of oil combined with a slowing oil demand expansion.} and (ii) to evaluate the consequences of a common monetary policy on inflation convergence and Price competitiveness.

The lack of a country-specific monetary policy would require further national policies (fiscal policies, structural reforms or regulations) in order to offset inflation differentials driven by oil price shocks. \cite{Castro2016} propose a methodology to evaluate paths of expected inflation conditional to different scenarios for the future price of oil by means of ARIMA models, transfer function models and fixed-interval smoother. The novelty of this work is the alternative approach to implement future scenarios for the highly volatile oil price movements and the intuitive way to deal with the transmission of the oil shocks on inflation. Quantifying the extent to which different likely oil price scenarios cause inflation differentials among European countries sheds light on the problem of ``one-size'' interest rate of the European Central Bank (ECB).

This investigation addresses three issues regarding common monetary policy responses to oil price fluctuations. First, the speed of convergence of inflation rates among EA members to the long term target and consequently the  monetary policy effectiveness. Second, concerns about the gain (or loss) of price competitiveness in countries with persistent inflation below (above) the inflation in the EA. Third, concerns about the impact of high (low) key interest rate to dampen (stimulate) the economic activity in countries with persistent inflation below (above) the inflation in the EA. In particular, the research seeks to answer the following questions:

\begin{itemize}
\item \textbf{What is the degree of inflation convergence across the euro area members after an oil price shock?} Countries with high oil price pass-through on inflation tend to drift away from the central bank objective of yearly inflation, weakening or strengthening inflation convergence. This issue can be assessed following, e.g., the procedure in \cite{Lopez2012}\footnote{Recently, other authors also have studied the inflation convergence in the EA. See, e.g., \cite{Becker2009}, \cite{Karanasos2016}, \cite{Ogrokhina2015}.} by evaluating the difference between each country's inflation and the cross sectional mean, as well as looking individual deviations from the group central tendency and from the ECB's benchmark.\footnote{In particular, they use panel unit root tests for convergence among series and find that the euro significantly anchors the individual and cross sectional volatility of the EA inflation rate, although Ireland, Greece and Spain reach higher rates than the other countries during the ``Great crisis''.} Based on future paths for inflation originated by several scenarios for oil price changes, it is our aim: (i) to evaluate the presence of unit roots for testing group-wise convergence;\footnote{The rejection of the null hypothesis of a unit root is interpreted as evidence that the series converge to their equilibrium state.} (ii) to calculate the strength of inflation group-wise convergence; or (iii) to estimate the half-life of the persistence of the inflation (the number of periods it takes for an oil price shock on the inflation differential to dissipate by 50 percent). Countries with non-rejection of the unit root null hypothesis (divergence), high rate of persistence of the differentials and larger half-lives caused by oil price shocks would need to implement more active fiscal policies to offset the effects of the shocks, in the absence of a country-specific monetary policy reaction.

\item \textbf{What is the loss of competitiveness within the EA members after an oil price shock?} Countries with higher than ECB's medium run rate of a 2\% inflation would suffer a loss in the relative price competitiveness. It is well known in the empirical literature that a rise in oil prices leads to a significant, direct and instantaneous increase in energy prices, especially in fuels for personal transport equipment (see, e.g., \citealp{Alvarez2011}). In contrast, many indirect and second-round effects on non-energy groups of inflation are limited. Indeed, \cite{Castro2016a}, considering higher level of disaggregation, found some significant differences in the effects on non-energy inflation among the main economies of the EA. Some of the effects are positive, explained by the reduction in the supply, while the negative effects may be associated with the slowdown in the demand.\footnote{Based on a VAR model, \cite{Castro2016b} find evidence that oil price pass-through to consumer prices is very low in general in the euro area as a whole, which suggests the adaptability of European producers to higher oil price pressures without transmitting them to consumers.} The final sign of the effect depends on the balance between the particular structure in the production and the idiosyncratic factor of consumption and consequently, we can find opposite effects on non-energy inflation in two different economies. Higher inflation in some goods or services might indicate a further loss of competitiveness due to indirect or second-round effects of oil price increases. When an increase in oil price leads to a rise in inflation for a good/service of one specific economy and/or a fall in the corresponding inflation for other economy, this may mean a competitiveness improvement for the latter economy. Based on future paths for inflation in the EA members originated by several scenarios for oil price changes, it is our goal: (i) to estimate the sensitivity of energy and non-energy inflation changes among European countries; and (ii) to quantify the corresponding changes in competitiveness within the EA members. Countries with persistent inflation above 2\% and sensitive to positive oil price shocks in relation to its partners suffer loss of competitiveness, requiring more active national policies (fiscal policies, structural reforms or regulations).

\item \textbf{Do EA members need further national policies to offset the effects of oil price shocks due to a lack of a country-specific monetary policy?} The monetary policy designed by the ECB takes into account the effects of oil price changes on EA inflation as a whole, despite the fact that they differ among countries.\footnote{Theoretically, the ECB responses to a positive oil price shock is a contractionary monetary policy (increase of short term interest rate, and dollar/euro and real effective exchange rates) while the CPI decreases.} Indeed, an oil price increase of 10\% accounts for 0.19 percentage points of the inflation in the EA, while the effects on inflation among countries range from 0.13 percentage points in Italy to 0.27 percentage points in Spain (see \citealp{Castro2016a}). On the other hand, the longest and sharpest oil price shocks have concentrated between 1996 and 2016. Following \cite{BlanchardGali2010}, we can identify 9 episodes of large oil price shocks during this period, from which 2 of them match up with peaks noticed in ECB key interest rate: (i) shock ended in 1999 with an annual oil price rate of 159.4\% and ECB key interest rate of 4.75\%; and (ii) shock ended in 2008 with a peak of oil price rate of 86.4\% and ECB key interest rate of 4.25\%.\footnote{The oil price growth is year-on-year in nominal terms and the ECB key interest rate is the ECB refinancing rate.}

Consequently, a common policy generates a situation where the key interest rate is too high for countries with low inflation and too low for countries with high inflation.\footnote{This could be related with the literature about asymmetries in the responses of single EA countries after common monetary shock (see, e.g., \citealp{Barigozzi2014}; \citealp{Cavallo2015}).} Put differently, real interest rates fall in countries with high inflation such as Spain, causing that government and household increasing their borrowing and spending.\footnote{The theoretical study of \cite{Aguiar2015} describes the impact of centralized monetary policy and decentralized fiscal policy of high-debt versus low-debt members, on debt dynamics and exposure to self-fulfilling debt crisis.} On the contrary, interest rates rise in countries with low inflation such as Germany and France slowing down real economic activity. Thus, ECB common monetary policy reaction to differential effects on inflation, driven by oil price shocks, would require setting additional and appropriate national fiscal policies, structural reforms or regulations. Based on future paths for inflation in the EA members originated by several scenarios for oil price changes, we can (i) calculate the appropriate national key interest rate; (ii) establish cluster of members where theoretical national key interest rate is significantly away ECB benchmark, therefore demanding further policy actions.

\end{itemize}

Summing up, the first phase of the proposed research consists on evaluating the sensitivity of inflation in the 19 euro area (EA) members to alternative scenarios about future oil prices. Building on these conditional inflation forecasts, the second phase focuses on quantifying the weakening or strengthening of inflation convergence among EA members. The third phase is devoted to analyze the relative loss of competitiveness within countries due to indirect or second-round effects of oil price increases. The fourth and final phase seeks to determine which countries would require further national policies in order to offset the oil price shocks, in the absence of a country-specific monetary policy reaction.

All the calculations will be implemented in \textbf{\textsf{R}} programming language and the reports written in {\LaTeX}. Excel will also be used as support for numerical inputs and outputs.

\bibliographystyle{apa}
\bibliography{ThesisBib}
\end{document}
